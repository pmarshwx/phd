%!TEX root = ../dissertation.tex


\chapter{Introduction}
\label{intro}

Rare meteorological events\footnote{\cite{Murphy1991} defined a rare meteorological event as one that occurs on less than five percent of forecasting occasions.} that occur on small spatial and short temporal scales pose significant challenges to forecasters. This is related to the limited predictability of phenomena occurring on short time-space scales; however, these events comprise a substantial portion of meteorological phenomena that negatively impact society (e.g., heavy rain, large hail, tornadoes, etc.). Thus, accurate numerical guidance of these events would provide large societal benefits.

Convection-allowing models (CAMs) have shown improved skill, compared to parameterized-convection models, in identifying regions where rare meteorological events associated with convection (hereafter RCEs\footnote{Rare Convective Event}) may occur \citep{Clark2010a}. Furthermore, CAMs are able to do this by explicitly representing deep-convective storms and their unique attributes --- not just storm environments \citep{Kain2010}. Yet, quantifying the uncertainty associated with explicit numerical prediction of RCEs is particularly challenging \citep{Sobash2011}. Of course, ensembles are powerful tools for quantifying uncertainty, but when convection-allowing ensemble prediction systems are used to provide guidance for forecasting storm-attributes, they are subject to the same fundamental limitation that handicaps single-member CAMS forecast systems: Too little is known about the performance characteristics of CAMs in predicting RCEs explicitly.

There are three main reasons for this deficiency. First, routine, explicit, contiguous  or near-contiguous United States (CONUS or near-CONUS) scale forecasts of RCEs have been available for only 6-7 years in the United States, so there is still much to learn about which phenomena can be skillfully predicted with convection-allowing models \citep{Kain2008, Kain2010}. Second, most real-time forecasting efforts with convection-allowing models have been short-term initiatives, focusing on specific tasks (e.g., \citealp{Done2004, Weisman2008}). Third, there is a limited database of forecasts for RCEs, making robust statistical techniques difficult (e.g., \citealp{Hamill2006}). In short, there is a limited track record in the use of CAMs as guidance for prediction of RCEs.

A strategy for calibrating, or quantifying the uncertainty of, forecasts of RCEs based on the idea of generating probabilistic forecasts from a single underlying deterministic model follows. This technique uses a conceptual approach similar to that described by \cite{Theis2005} and refined by \cite{Sobash2011}. As in these two studies, this strategy differs from other methods for both deterministic models (e.g., \citealp{Glahn1972}) and ensemble modeling systems (e.g., \citealp{Hamill1998, Raftery2005, Clark2009, Glahn2009}) by including a neighborhood around each model grid point as a fundamental component of the calibration process.

This strategy is rooted in the fundamental concepts of Kernel Density Estimation (KDE), which can be used to retrieve spatial probability distributions from point observations, or, in this case, forecasts. In other words, if a model forecasts an event at point A, KDE can be utilized to gain insight into the probability that the event might occur at a nearby point. This is achieved by utilizing a statistical distribution to redistribute the total probability (100\%) from point A over multiple (typically nearby) grid points. The result is a probability forecast, the character of which is determined by one's choice of statistical distribution and the number of grid points over which the distribution is applied. The resulting smoothing effect is similar to that obtained with ensemble output by \cite{Wilks2002}, but initial calibration efforts herein focus on output from a single deterministic model. \cite{Sobash2011} demonstrated with a two-dimensional, isotropic Gaussian function that calibration of the probability forecasts derived using this technique is most easily done by changing the number of grid points over which non-zero probabilities are distributed. Here, however, an objective calibration method based on past model performance, is presented.

The method and initial results are put forth in the following chapters. Chapter \ref{method} provides a description of the method, chapter \ref{data} provides a description of the data, and chapter \ref{deterministic} provides the results for a single, deterministic model. Chapter \ref{ensemble} provides discussion and results on how to extend this method from a single, deterministic model framework to an ensemble framework.


