%!TEX root = ../dissertation.tex


\section{Ensemble Results}
\label{eresults}




\subsection{25.4 mm Threshold}
\label{eresults_25.4mm}

As previously mentioned, the first step of the proposed calibration process is to create two-dimensional composites of where observations occurred relative to forecast grid points.
In the case of the small number of SSEF forecasts, this amounts to creating a composite for each member for each simulation.
\mbox{Figure \ref{ssef-25mm-400km-composite}} shows the the two-dimensional composites for each member at the \mbox{25.4 mm} in \mbox{6 hr} threshold for one of the twenty simulations.
In this simulation, it is readily apparent that the WRF-NMM ensemble members are generaly the wettest, with the WRF-ARW members drier.
(The ARPS member is in between.)
The overall axis of observations relative to forecasts indicates a general southeast-to-northeast orientation in most members, although this is more readily apparent in some members than others.


Although the orientation of the axis of maximum observations tends to generally be similar between members, the centroid of the distribution exhibits more variability.
About half of the members have the centroid of observations too far east and half having the centroid be too far west.
(This corresponds to the $h$ parameter of the Gaussian fitting parameters.)
The members having the observations centroid too far east tended to be farther away from the forecast than those members having the observations centroid too far west.
In this simulation, the WRF-NMM has a westward bias with its forecasts, as the centroid of observations is east of the forecast grid point in every WRF-NMM member.
A similar observation cannot be made for the WRF-ARW members.
It is unclear from this single simulation if the westward forecast bias in the WRF-NMM members is systematic of the WRF-NMM core, or merely a function of only having 4 WRF-NMM members.


When examining the north/south variations of the observations centroid relative to the model forecast, similar variations are observed.
(The north/south displacement of the observations centroid corresponds to the $k$ parameter of the Gaussian fitting parameters.)
Slightly more than half the members have the centroid of observations too far north, indicating a southward bias of the forecast, with slightly less than half having the centroid of observations too far south.
Three of the four WRF-NMM members had the forecast too far north, with the WRF-NMM control member having the greatest displacement.
No obvious preference in displacement direction is readily apparent from the WRF-ARW members.
However, it does appear that generally speaking, the magnitude of the displacement of the WRF-ARW members appears to be less than that of the WRF-NMM members.


The lower right panel of \mbox{Figure \ref{ssef-25mm-400km-composite}} depicts the standard deviation between the two-dimensional composites of each member for the given simulation.
In this panel, the darker colors indicate a greater standard deviation at that particular grid point than grid points with a lighter color.
It is readily apparent that the greatest variability between the members exists to the east of the forecast grid point.
This is due to the wetter WRF-NMM members having a westward forecast bias, resulting in a wider range of observation counts to the east of the forecast.


Unfortunately, \mbox{Figure \ref{ssef-25mm-400km-composite}} is only one simulation out of the twenty simulations produced.
In an attempt to gain insights into the variation of the two-dimensional composites for each member between all twenty simulations, the standard deviation of each member's composites is shown in \mbox{Figure \ref{ssef-25mm-400km-std}}.
A cursory examination of the standard deviation of the two-dimensional composites does not yield any immediate insights.
Some members exhibit a maximum in variability along a southwest to northeast orientation.
Other members exhibit a more uniform increase in the maximum variability.
In both cases, the maximum variability tended to be located near the forecast grid point, with generally decreasing variability as distance from the forecast grid point increases.


Breaking down the fitting parameters by simulation gives an even better idea of the variability in each member's twenty two-dimensional composites.
A set of five figures (\mbox{Figures \ref{sigmax-25mm-400km-dist}--\ref{k-25mm-400km-dist}}) were created to examine the variability of each of the five Gaussian fitting parameters --- one figure per fitting parameter.
The figures contain box-and-whisker plots for each member's distribution of that figure's fitting parameter.
The box-and-whisker plots offer insight into the variability in the various fitting functions used by each member.
In each of the figures, the red horizontal line marks the median value of the distributions, and the blue box represents the 25th and 75th percentiles.
The whiskers denote the range of the distribution, up to $\pm$ 1.5 times the interquartile range.
Gold stars are used to denote any outliers, defined to be any value of the distribution that is outside the range of $\pm$ 1.5 times the interquartile range.


\mbox{Figure \ref{sigmax-25mm-400km-dist}} displays the box-and-whisker plots for the $\sigma_x$ fitting parameter.
This parameter is the length of the long axis of the fitted anisotropic Gaussian function.
Nine out of the fifteen ensemble members do not have outliers.
Of the remaining six ensemble members with outliers, two of them only have one outlier, two of them have two outliers, and one each has four and five outliers.
Ensemble members that have outliers had a range of over 20 kilometers for $\sigma_x$, whereas those members without outliers generally had ranges of less than 20 kilometers.
This means that 40\% of the ensemble members had variability in their $\sigma_x$ parameter that was greater than 10\% of the median length of the $\sigma_x$ fitting parameter.


\mbox{Figure \ref{sigmay-25mm-400km-dist}} shows the box-and-whisker plots for the $\sigma_y$ fitting parameter.
This parameter is the length of the short axis of the fitted anisotropic Gaussian function.
Unlike with the $\sigma_x$ fitting parameter, the ensemble member distributions of the $\sigma_y$ fitting parameter exhibit fewer outliers with only four members having them.
This can be attributed to their being more variability in the 25th to 75th percentile, as noted by the increased size of the boxes for many members.
Furthermore, whereas when the maximum range of the $\sigma_x$ fitting parameter distribution greater than 20 kilometers indicated the presence of an outlier, several members have distributions of $\sigma_y$ with a maximum range greater than 20 kilometers without having an outlier.
In fact, six out of the ten WRF-ARW members have a range of $\sigma_y$ greater than 20 kilometers and only two of them have outliers.


The distributions of the counter-clockwise rotation angle of the abscissa, fitting parameter $\theta$, is shown in \mbox{Figure \ref{theta-25mm-400km-dist}}.
For this fitting parameter, eleven out of fifteen members exhibit outliers.
This is not really surprising when one considers the overlap in the distributions of fitting parameters $\sigma_x$ and $\sigma_y$.
As alluded to in \mbox{Section \ref{std}}, when $\sigma_x$ approaches $\sigma_y$, the Gaussian function approaches being isotropy.
As a Gaussian function approaches isotropy the variability of the rotation angle, $\theta$ increases as a result of the more circular nature to the function.
It is much more difficult to accurately measure the orientation of the rotation angle of an ellipse that is nearly circular than that of one with a high level of eccentricity.


The variability in the fitting parameters was not limited to just the shape of the anisotropic Gaussian.
The location of the fitting Gaussian also ranged widely.
Each member's distributions of the $h$ fitting parameter (displacement in the east or west direction) is shown in \mbox{Figure \ref{h-25mm-400km-dist}}.
Generally speaking, some of the values for $h$ deviated from the forecast grid point by over 40 kilometers.
The ARPS and WRF-NMM members consistently demonstrated a bias toward the observations centroid being east.
This corresponds to the forecasts, on average, being too far west compared to where observations occurred.
Most of the WRF-ARW members exhibited the opposite behavior, namely having the observations being too far east (indicating an eastward bias with the forecast).
Not every WRF-ARW member exhibited this bias, however.
The values comprising the interquartile range for two of the WRF-ARW members consisted of positive values for $h$, indicating a westward forecast bias.
However, unlike with the ARPS and WRF-NMM members, where the entire distribution consisted of positive values for $h$, every member of the WRF-ARW had at least a portion of the ``whiskers'' part of the box-and whiskers plot with negative values, indicating an eastward forecast bias, as compared to observations.













