%!TEX root = ../dissertation.tex


\chapter{Ensemble}
\label{ensemble}



%!TEX root = ../dissertation.tex


\section{Ensemble Data}
\label{edata}




\subsection{Ensemble Forecasts}
\label{emodel}

As discussed at the beginning of \mbox{chapter \ref{deterministic}}, in order to develop a statistical post-processing method for calibration of numerical weather prediction forecasts, both forecasts and observations must be readily available.
Unfortunately, as mentioned in the Introduction, ``\dots there is a limited database of forecasts for RCEs, making robust statistical techniques difficult\dots''
As limited as databases of deterministic CAM forecasts are, the availability of ensembles of CAM forecasts is even worse.


Similar to CAM forecasts, most ensemble CAM (eCAM) forecasts have been short-term initiatives, produced in support of various field programs and experiments.
The makeup of these eCAMs are often modified year-to-year to adapt to the changing needs of the various field programs and experiments.
This makes finding a long running, consistent ensemble extremely difficult.


One of the largest collections of eCAM forecasts has been produced by the University of Oklahoma's Center for the Analysis and Prediction of Storms (CAPS).
Since 2007, CAPS has been producing CAM forecasts in support of the Hazardous Weather Testbed's (HWT) annual Spring Forecast Experiment (SFE).
From 2007 through 2010, the ensemble configuration changed from year-to-year based on the results of previous years.
However, in 2011 a conscious effort was made to ensure a subset of that year's CAM forecasts would be produced using as similar a configuration as the previous year.


In 2010, CAPS produced a 26 member ensemble of convection-allowing model forecasts.
This eCAM was multi-model in nature, initialized daily at 00 UTC, used 4 km grid spacing, and integrated out to 30 hours.
Of the 26 members, 19 were WRF-ARW\footnote{Weather Research and Forecasting -- Advanced Research WRF} \citep{WRFV3}, 5 were WRF-NMM\footnote{Weather Research and Forecasting -- nonhydrostatic Mesoscale Model} \citep{NAMnWRF-NMM}, and 2 were ARPS\footnote{Advanced Regional Prediction System} \citep{ARPS}. In 2011, CAPS produced a 50 member ensemble consisting of 41 WRF-ARW, 5 WRF-NMM, and 4 ARPS forecasts initialized daily at 00 UTC, using a grid spacing of 4 km, and integrated forward to 36 hours.


From these 26 members in 2010 and 50 members in 2011, 15 were chosen to be held as consistent as possible between the two years.
These 15 members were composed of 10 WRF-ARW forecasts, 4 WRF-NMM forecasts, and 1 ARPS forecast.
The background initial conditions for these 15 members were downscaled from the 00 UTC 12 km NAM, with additional information coming from a three dimensional variational and cloud analysis from ARPS.
Except for the control members (one each from the WRF-ARW, WRF-NMM, and ARPS), these initial conditions were then perturbed using mesoscale atmospheric perturbations from NOAA's Environmental Modeling Center's operational Short-Range Ensemble Forecast (SREF) system.
Boundary conditions for the three control members came from the 00 UTC NAM's forecasts, whereas the remaining 12 members used the SREF forecast corresponding to the perturbations used in the initial conditions. A listing of the configurations for each member of the eCAM can be found in \mbox{Table \ref{ensemble_members}}.


The only controlled change between 2010 and 2011 for the aforementioned set of 15 forecasts came from a change in the version of WRF. In 2010, \mbox{WRF Version 3.1.1} was used whereas \mbox{WRF Version 3.2.1} was used in 2011.
Changes in the NAM and SREF, which were used for initial and boundary conditions were controlled by the NOAA National Weather Service and could not be avoided.




\subsection{Observations}
\label{eobservations}

As was the case for the deterministic model calibration, NCEP's Stage IV national quantitative precipitation estimate analysis was chosen as the verification dataset\footnote{See \mbox{section \ref{observations}} for a description of this dataset.}.
Similarly, as was necessary with the deterministic forecasts, the ensemble forecasts have been interpolated to the Stage IV grid, and all diagnostics and analysis were conducted on this grid.
Furthermore, the same mask shown in \mbox{Figure \ref{domain}} was used to limit the analysis to areas east of the Rocky Mountains and near land.




\subsection{Processing and Data Selection}
\label{eprocessing}

CAPS produced ensemble forecasts for the 2010 HWT SFE over a five week period from 17 May through 18 June.
In 2011 CAPS produced ensemble forecasts for the HWT SFE from 09 May through 10 June.
CAPS produced two retrospective forecasts in 2011 for the 27 April 2011 tornado outbreak in the southeast United States and the 22 May 2011 Joplin, Missouri EF-5 tornado.
\mbox{Table \ref{sfedtes}} gives the exact list of dates CAPS forecasts are available.




%!TEX root = ../dissertation.tex


\section{Ensemble Results}
\label{eresults}







% table definitions
%%%%%%%%%%%%%%%%%%%
\newenvironment{telement}[2]{\begin{minipage}[c]{#1} \centering #2} {\end{minipage}}
\newcommand{\coremem}{\color{black}}
\newcommand{\member}[1]{\begin{telement}{0.75in}{#1}\end{telement}}
\newcommand{\ic}[1]{\begin{telement}{1in}{#1}\end{telement}}
\newcommand{\bc}[1]{\begin{telement}{0.75in}{#1}\end{telement}}
\newcommand{\radar}[1]{\begin{telement}{0.75in}{#1}\end{telement}}
\newcommand{\microphysics}[1]{\begin{telement}{0.75in}{#1}\end{telement}}
\newcommand{\lsm}[1]{\begin{telement}{0.75in}{#1}\end{telement}}
\newcommand{\pbl}[1]{\begin{telement}{0.75in}{#1}\end{telement}}

\newpage

\begin{center}
    \renewcommand{\arraystretch}{3}
    \centering
    \singlespace
    \small
    \setlength\tabcolsep{2pt}
    \rowcolors{2}{gray!20}{white}
    \begin{longtable}{|c|c|c|c|c|c|c|}
        \caption[Configurations for the 2010 and 2011 CAPS Ensemble Members]
        {Configurations for the 2010 and 2011 CAPS Ensemble Members.}
        \label{ensemble_members} \\

        % Header for the first page
        \hline
        \rowcolor{gray!60}
        \member{\textbf{Member}} &
        \ic{\textbf{I. C.}} &
        \bc{\textbf{B. C.}} &
        \radar{\textbf{Radar}} &
        \microphysics{\textbf{Micro-\\physics}} &
        \lsm{\textbf{LSM}} &
        \pbl{\textbf{PBL}} \\
        \hline
        \endfirsthead

        % Header for remaining pages
        \hline
        \multicolumn{7}{|c|}{{\tablename} \thetable{} -- Continued} \\
        \hline
        \rowcolor{gray!60}
        \member{\textbf{Member}} &
        \ic{\textbf{I. C.}} &
        \bc{\textbf{B. C.}} &
        \radar{\textbf{Radar}} &
        \microphysics{\textbf{Micro-\\physics}} &
        \lsm{\textbf{LSM}} &
        \pbl{\textbf{PBL}} \\
        \hline
        \endhead

        %This is the footer for all pages except the last page of the table...
        \multicolumn{7}{|c|}{{Continued on Next Page \ldots}} \\
        \hline
        \endfoot

        %This is the footer for the last page of the table...
        \hline
        \multicolumn{7}{|l|}{Note 1: For all members, cumulus parameterization is turned off} \\
        \hline
        \multicolumn{7}{|l|}{\multirow{1}{\textwidth}{Note 2: For all ARW members, the long-wave radiation parameterization is RRTM and the short-wave radiation parameterization is Goddard}} \\
        \hline
        \multicolumn{7}{|l|}{\multirow{1}{\textwidth}{Note 3: For nmm\_cn, nmm\_m2, \& nmm\_m3 the long-wave radiation parameterization is GFDL and the short-wave radiation parameterization is GFDL}} \\
        \hline
        \multicolumn{7}{|l|}{\multirow{1}{\textwidth}{Note 4: For nmm\_m4 \& nmm\_m5 the long-wave radiation parameterization is RRTM and the short-wave radiation parameterization is Dudhia}}\\
        \hline
        \multicolumn{7}{|l|}{Note 5: The arps member uses Chou/Suarex for radiation} \\
        \hline
        \multicolumn{7}{|l|}{Note 6: Ferrier+ refers to a subset of changes in the updated version now in NEMS/NMMB} \\
        \hline
        \multicolumn{7}{|l|}{\multirow{1}{\textwidth}{Note 7: The ARPS PBL scheme \citep{Xue1996, Sun1986} uses a non-local vertical mixing length within the convective boundary layer}} \\
        \hline
        \endlastfoot

        % Member 1 of 24
        \hline
        \coremem\member{arw\_cn} &
        \coremem\ic{00Z ARPS 3DVAR \& Cloud Analysis} &
        \coremem\bc{00Z NAM Forecast} &
        \coremem\radar{Yes} &
        \coremem\microphysics{Thompson} &
        \coremem\lsm{Noah} &
        \coremem\pbl{MYJ} \\

        % Member 2 of 24
        \hline
        \coremem\member{arw\_m4} &
        \coremem\ic{arw\_cn + em\_p1\_pert} &
        \coremem\bc{21Z SREF em\_p1} &
        \coremem\radar{Yes} &
        \coremem\microphysics{Morrison} &
        \coremem\lsm{RUC} &
        \coremem\pbl{YSU} \\

        % Member 3 of 24
        \hline
        \coremem\member{arw\_m5} &
        \coremem\ic{arw\_cn + em\_p2\_pert} &
        \coremem\bc{21Z SREF em\_p2} &
        \coremem\radar{Yes} &
        \coremem\microphysics{Thompson} &
        \coremem\lsm{Noah} &
        \coremem\pbl{QNSE} \\

        % Member 4 of 24
        \hline
        \coremem\member{arw\_m6} &
        \coremem\ic{arw\_cn - nmm\_p1\_pert} &
        \coremem\bc{21 SREF nmm\_p1} &
        \coremem\radar{Yes} &
        \coremem\microphysics{WSM6} &
        \coremem\lsm{RUC} &
        \coremem\pbl{QNSE} \\

        % Member 5 of 24
        \hline
        \coremem\member{arw\_m7} &
        \coremem\ic{arw\_cn + nmm\_p2\_pert} &
        \coremem\bc{21Z SREF nm\_p2} &
        \coremem\radar{Yes} &
        \coremem\microphysics{WDM6} &
        \coremem\lsm{Noah} &
        \coremem\pbl{MYNN} \\

        % Member 6 of 24
        \hline
        \coremem\member{arw\_m8} &
        \coremem\ic{arw\_cn + rsm\_n1\_pert} &
        \coremem\bc{21Z SREF rsm\_n1} &
        \coremem\radar{Yes} &
        \coremem\microphysics{Ferrier} &
        \coremem\lsm{RUC} &
        \coremem\pbl{YSU} \\

        % Member 7 of 24
        \hline
        \coremem\member{arw\_m9} &
        \coremem\ic{arw\_cn - etaKF\_n1\_pert} &
        \coremem\bc{21Z SREF etaKF\_n1} &
        \coremem\radar{Yes} &
        \coremem\microphysics{Ferrier} &
        \coremem\lsm{Noah} &
        \coremem\pbl{YSU} \\

        % Member 8 or 24
        \hline
        \coremem\member{arw\_m10} &
        \coremem\ic{arw\_cn + etaKF\_p1\_pert} &
        \coremem\bc{21Z SREF etaKF\_p1} &
        \coremem\radar{Yes} &
        \coremem\microphysics{WDM6} &
        \coremem\lsm{Noah} &
        \coremem\pbl{QNSE} \\

        % Member 9 of 24
        \hline
        \coremem\member{arw\_m11} &
        \coremem\ic{arw\_cn - etaBMJ\_p1\_pert} &
        \coremem\bc{21Z SREF etaBMJ\_p1 } &
        \coremem\radar{Yes} &
        \coremem\microphysics{WSM6} &
        \coremem\lsm{RUC} &
        \coremem\pbl{MYNN} \\

        % Member 10 of 24
        \hline
        \coremem\member{arw\_m12} &
        \coremem\ic{arw\_cn + etaBMJ\_p1\_pert} &
        \coremem\bc{21Z SREF etaBMJ\_p1} &
        \coremem\radar{Yes} &
        \coremem\microphysics{Thompson} &
        \coremem\lsm{RUC} &
        \coremem\pbl{MYNN} \\

        % Member 19 of 24
        \hline
        \coremem\member{nmm\_cn} &
        \coremem\ic{00Z ARPS 3DVAR \& Cloud Analysis} &
        \coremem\bc{00Z NAM Forecast} &
        \coremem\radar{Yes} &
        \coremem\microphysics{Ferrier} &
        \coremem\lsm{Noah} &
        \coremem\pbl{MYJ} \\

        % Member 21 of 24
        \hline
        \coremem\member{nmm\_m3} &
        \coremem\ic{nmm\_cn + nmm\_n1\_pert} &
        \coremem\bc{21Z SREF nmm\_n1} &
        \coremem\radar{Yes} &
        \coremem\microphysics{Thompson} &
        \coremem\lsm{Noah} &
        \coremem\pbl{MYJ} \\

        % Member 22 of 24
        \hline
        \coremem\member{nmm\_m4} &
        \coremem\ic{arw\_cn + nmm\_n2\_pert} &
        \coremem\bc{21Z SREF nmm\_n2} &
        \coremem\radar{Yes} &
        \coremem\microphysics{WSM6} &
        \coremem\lsm{RUC} &
        \coremem\pbl{MYJ} \\

        % Member 23 of 24
        \hline
        \coremem\member{nmm\_m5} &
        \coremem\ic{arw\_cn + em\_n1\_pert} &
        \coremem\bc{21Z SREF em\_n1} &
        \coremem\radar{Yes} &
        \coremem\microphysics{Ferrier} &
        \coremem\lsm{RUC} &
        \coremem\pbl{MYJ} \\

        % Member 24 or 24
        \hline
        \coremem\member{arps\_cn} &
        \coremem\ic{00Z ARPS 3DVAR \& Cloud Analysis} &
        \coremem\bc{00Z NAM Forecast} &
        \coremem\radar{Yes} &
        \coremem\microphysics{Lin} &
        \coremem\lsm{Force Restore} &
        \coremem\pbl{1.5-order TKE-based} \\
        \hline

    \end{longtable}
\end{center}




\begin{table}[cc]
    \centering
    \setlength{\tabcolsep}{0.5in}
    \renewcommand{\arraystretch}{1.25}
    \caption[2010 and 2011 dates where CAPS forecasts are available.]
    {2010 and 2011 dates where CAPS forecasts are available.}
    \label{sfedtes} \vspace{0.25in}
    \begin{tabular}{||c c||}
        \hline \hline
        \vspace{\baselineskip} & \vspace{\baselineskip} \\
        \multicolumn{2}{||c||}{\Large 2010 Dates} \\
        \vspace{\baselineskip} & \vspace{\baselineskip} \\
        Week \#1: & 17--21 May 2010 \\
        Week \#2: & 24--28 May 2010 \\
        Week \#3: & 31 May -- 04 June 2010 \\
        Week \#4: & 07--11 June 2010 \\
        Week \#5: & 14--18 June 2010 \\
        \vspace{\baselineskip} & \vspace{\baselineskip} \\
        \hline
        \vspace{\baselineskip} & \vspace{\baselineskip} \\
        \multicolumn{2}{||c||}{\Large 2011 Dates} \\
        \vspace{\baselineskip} & \vspace{\baselineskip} \\
        Week \#1: & 09--13 May 2011 \\
        Week \#2: & 16--20 May 2011 \\
        Week \#3: & 23--27 May 2011 \\
        Week \#4: & 30 May -- 03 June 2011 \\
        Week \#5: & 06--10 June 2011 \\
        \vspace{\baselineskip} & \vspace{\baselineskip} \\
        \hline
        \vspace{\baselineskip} & \vspace{\baselineskip} \\
        \multicolumn{2}{||c||}{\Large Special Run Dates} \\
        \vspace{\baselineskip} & \vspace{\baselineskip} \\
        Special \#1: & 27 April 2011 \\
        Special \#2: & 22 May 2011 \\
        \vspace{\baselineskip} & \vspace{\baselineskip} \\
        \hline \hline
    \end{tabular}
\end{table}




% CHAPTER FIGURES HERE --- Ensures they all show up at end of chapter

\clearpage
\begin{figure}[cc]
    \centering
    \includegraphics[width=\textwidth, height=\textheight, keepaspectratio]{%
    ./ensemble/figs/ssef_25mm_400km_sigmax.png}\\
    \caption{Box-and-Whisker plots of the $\sigma_x$ anisotropic Gaussian fitting parameter, for each member of the SSEF at the \mbox{25.4 mm} in \mbox{6 hr} threshold.
    Each SSEF member's distribution is derived from the twenty re-sampled simulations.}
    \label{sigmax-25mm-400km-dist}
\end{figure}


\clearpage
\begin{figure}[cc]
    \centering
    \includegraphics[width=\textwidth, height=\textheight, keepaspectratio]{%
    ./ensemble/figs/ssef_25mm_400km_sigmay.png}\\
    \caption{The same as in \mbox{Figure \ref{sigmax-25mm-400km-dist}}, except for fitting parameter $\sigma_y$.}
    \label{sigmay-25mm-400km-dist}
\end{figure}


\clearpage
\begin{figure}[cc]
    \centering
    \includegraphics[width=\textwidth, height=\textheight, keepaspectratio]{%
    ./ensemble/figs/ssef_25mm_400km_theta.png}\\
    \caption{The same as in \mbox{Figure \ref{sigmax-25mm-400km-dist}}, except for fitting parameter $\theta$.}
    \label{theta-25mm-400km-dist}
\end{figure}


\clearpage
\begin{figure}[cc]
    \centering
    \includegraphics[width=\textwidth, height=\textheight, keepaspectratio]{%
    ./ensemble/figs/ssef_25mm_400km_h.png}\\
    \caption{The same as in \mbox{Figure \ref{sigmax-25mm-400km-dist}}, except for fitting parameter $h$.}
    \label{h-25mm-400km-dist}
\end{figure}


\clearpage
\begin{figure}[cc]
    \centering
    \includegraphics[width=\textwidth, height=\textheight, keepaspectratio]{%
    ./ensemble/figs/ssef_25mm_400km_k.png}\\
    \caption{The same as in \mbox{Figure \ref{sigmax-25mm-400km-dist}}, except for fitting parameter $k$.}
    \label{k-25mm-400km-dist}
\end{figure}


\clearpage
\begin{figure}[cc]
    \centering
    \includegraphics[width=\textwidth, height=\textheight, keepaspectratio]{%
    ./ensemble/figs/ssef_25mm_400km_composite_std.png}\\
    \caption{Plots of the standard deviation of the number of observations at each grid point, for each member of the SSEF at the \mbox{25.4 mm} in \mbox{6 hr} threshold.
    The standard deviation represents variability in the number of observational counts at each grid point between each of the twenty simulations.
    The lower right panel is a depiction of the standard deviation at each grid point for all members and all simulations.
    The color scale is different than all other panels to indicate the scale for this panel is different from the scale for all other members.
    A scale for the lower right panel is not shown as its purpose is to be qualitative instead of quantitative.}
    \label{ssef-25mm-400km-std}
\end{figure}


\clearpage
\begin{figure}[cc]
    \centering
    \includegraphics[width=\textwidth, height=\textheight, keepaspectratio]{%
    ./ensemble/figs/ssef_25mm_400km_composite_example.png}\\
    \caption{Plots of the two-dimensional composites at the \mbox{25.4 mm} in \mbox{6 hr} threshold for each member of the SSEF for a specific simulation.
    The lower right panel qualitatively depicts the standard deviation at each grid point for each of the two-dimensional composites shown in the other panels.
    As was the case in \mbox{Figure \ref{ssef-25mm-400km-std}}, the color scale is different from the others to prevent comparison with the other panels.}
    \label{ssef-25mm-400km-composite}
\end{figure}


\clearpage
\begin{figure}[cc]
    \centering
    \includegraphics[width=\textwidth, height=\textheight, keepaspectratio]{%
    ./ensemble/figs/ssef_25mm_400km_example.png}\\
    \caption{Example probabilistic forecasts of exceeding \mbox{25.4 mm} in \mbox{6 hr} from each SSEF member for the six hours ending 06 UTC 20 May 2010.
    The Stage IV QPE greater than \mbox{25.4 mm} is contoured on top of the probability forecasts for each member.
    The lower right panel depicts the ensemble average probability.}
    \label{ssef-25mm-400km-example}
\end{figure}


\clearpage
\begin{figure}[cc]
    \centering
    \includegraphics[width=\textwidth, height=\textheight, keepaspectratio]{%
    ./ensemble/figs/ssef_25mm_400km_verif.png}\\
    \caption{Performance Diagram (a) and reliability diagram (b) for each member of the SSEF for the \mbox{25.4 mm} in \mbox{6 hr}.
    The line of perfect reliability (diagonal; dashed) is also plotted on the reliability diagram.
    The mean values for each individual SSEF members' twenty simulations are shown in black, with standard errors in light gray.
    The ensemble average forecast verification is shown in blue.
    The red curve depicts the verification of the modified Hamill and Colucci porbabilistic forecasts.
    }
    \label{ssef-25mm-400km-verif}
\end{figure}


% 12.7 mm Figures


\clearpage
\begin{figure}[cc]
    \centering
    \includegraphics[width=\textwidth, height=\textheight, keepaspectratio]{%
    ./ensemble/figs/ssef_12mm_400km_sigmax.png}\\
    \caption{The same as in \mbox{Figure \ref{sigmax-25mm-400km-dist}}, except for threshold \mbox{12.7 mm} in \mbox{6 hr}.}
    \label{sigmax-12mm-400km-dist}
\end{figure}


\clearpage
\begin{figure}[cc]
    \centering
    \includegraphics[width=\textwidth, height=\textheight, keepaspectratio]{%
    ./ensemble/figs/ssef_12mm_400km_sigmay.png}\\
    \caption{The same as in \mbox{Figure \ref{sigmay-25mm-400km-dist}}, except for threshold \mbox{12.7 mm} in \mbox{6 hr}.}
    \label{sigmay-12mm-400km-dist}
\end{figure}


\clearpage
\begin{figure}[cc]
    \centering
    \includegraphics[width=\textwidth, height=\textheight, keepaspectratio]{%
    ./ensemble/figs/ssef_12mm_400km_theta.png}\\
    \caption{The same as in \mbox{Figure \ref{theta-25mm-400km-dist}}, except for threshold \mbox{12.7 mm} in \mbox{6 hr}.}
    \label{theta-12mm-400km-dist}
\end{figure}


\clearpage
\begin{figure}[cc]
    \centering
    \includegraphics[width=\textwidth, height=\textheight, keepaspectratio]{%
    ./ensemble/figs/ssef_12mm_400km_h.png}\\
    \caption{The same as in \mbox{Figure \ref{h-25mm-400km-dist}}, except for threshold \mbox{12.7 mm} in \mbox{6 hr}.}
    \label{h-12mm-400km-dist}
\end{figure}


\clearpage
\begin{figure}[cc]
    \centering
    \includegraphics[width=\textwidth, height=\textheight, keepaspectratio]{%
    ./ensemble/figs/ssef_12mm_400km_k.png}\\
    \caption{The same as in \mbox{Figure \ref{k-25mm-400km-dist}}, except for threshold \mbox{12.7 mm} in \mbox{6 hr}.}
    \label{k-12mm-400km-dist}
\end{figure}


\clearpage
\begin{figure}[cc]
    \centering
    \includegraphics[width=\textwidth, height=\textheight, keepaspectratio]{%
    ./ensemble/figs/ssef_12mm_400km_composite_std.png}\\
    \caption{The same as in \mbox{Figure \ref{ssef-25mm-400km-std}}, except for threshold \mbox{12.7 mm} in \mbox{6 hr}.}
    \label{ssef-12mm-400km-std}
\end{figure}


\clearpage
\begin{figure}[cc]
    \centering
    \includegraphics[width=\textwidth, height=\textheight, keepaspectratio]{%
    ./ensemble/figs/ssef_12mm_400km_composite_example.png}\\
    \caption{The same as in \mbox{Figure \ref{ssef-25mm-400km-composite}}, except for threshold \mbox{12.7 mm} in \mbox{6 hr}.}
    \label{ssef-12mm-400km-composite}
\end{figure}


\clearpage
\begin{figure}[cc]
    \centering
    \includegraphics[width=\textwidth, height=\textheight, keepaspectratio]{%
    ./ensemble/figs/ssef_12mm_400km_example.png}\\
    \caption{The same as in \mbox{Figure \ref{ssef-25mm-400km-example}}, except for threshold \mbox{12.7 mm} in \mbox{6 hr}.}
    \label{ssef-12mm-400km-example}
\end{figure}


\clearpage
\begin{figure}[cc]
    \centering
    \includegraphics[width=\textwidth, height=\textheight, keepaspectratio]{%
    ./ensemble/figs/ssef_12mm_400km_verif.png}\\
    \caption{The same as in \mbox{Figure \ref{ssef-25mm-400km-verif}}, except for threshold \mbox{12.7 mm} in \mbox{6 hr}.}
    \label{ssef-12mm-400km-verif}
\end{figure}


% HPC Verification Scores


\clearpage
\begin{figure}[cc]
    \centering
    \includegraphics[width=\textwidth, height=\textheight, keepaspectratio]{%
    ./ensemble/figs/hpc1218.png}\\
    \caption{Quantitative precipitation forecast verification scores at the \mbox{12.7 mm} in \mbox{6 hr} threshold for the GFS and NAM numerical models and the HPC human forecasts.
    This is for forecasts with lead time of \mbox{12-18 hr}.}
    \label{hpc18}
\end{figure}


\clearpage
\begin{figure}[cc]
    \centering
    \includegraphics[width=\textwidth, height=\textheight, keepaspectratio]{%
    ./ensemble/figs/hpc1824.png}\\
    \caption{The same as in \mbox{Figure \ref{hpc18}}, except for forecasts with lead time of \mbox{18-24 hr}.}
    \label{hpc24}
\end{figure}


\clearpage
\begin{figure}[cc]
    \centering
    \includegraphics[width=\textwidth, height=\textheight, keepaspectratio]{%
    ./ensemble/figs/hpc2430.png}\\
    \caption{The same as in \mbox{Figure \ref{hpc18}}, except for forecasts with lead time of \mbox{24-30 hr}.}
    \label{hpc30}
\end{figure}


\clearpage
\begin{figure}[cc]
    \centering
    \includegraphics[width=\textwidth, height=\textheight, keepaspectratio]{%
    ./ensemble/figs/hpc3036.png}\\
    \caption{The same as in \mbox{Figure \ref{hpc18}}, except for forecasts with lead time of \mbox{30-36 hr}.}
    \label{hpc36}
\end{figure}



