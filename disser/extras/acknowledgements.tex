%!TEX root = ../dissertation.tex


\begin{quotation}
    \textit{
        ``We are like dwarfs sitting on the shoulders of giants.
        We see more, and things that are more distant, than they did, not because our sight is superior or because we are taller than they, but because they raise us up, and by their great stature add to ours.''
    }
    \begin{flushright}
        --- John of Salisbury, \emph{Metalogicon} (1159)
    \end{flushright}
\end{quotation}


You do not simply write a dissertation overnight.
It takes many years, or in my case, a lifetime, of people pouring themselves into you.
People who give selflessly of their time and energy to ensure that you achieve your goal(s).
So when it comes time to finish that endeavor, how do you recognize everyone that has played a role in helping you get to the point of writing a dissertation?
How can a few typed words convey the lifetime of appreciation and thanks that everyone so rightly deserves?
I know that words can never fully convey the appreciation I have for everyone who has helped me along my journey, but I offer them up as an attempt.


First and foremost I must thank my main advisor, Dr. John (Jack) Kain for taking a chance on a graduate student he barely knew.
Without his willingness to rescue me from myself, I would not be writing a dissertation, nor would I be currently employed at the Storm Prediction Center.
Furthermore without his guidance and patience, and not to mention willingness to let me explore scientific endeavors not directly related to my dissertation research, I would not be poised on the threshold of entering the community of scholars.
I am forever indebted to him.
Along the same lines, thanks must be given to my co-advisor, Dr. Kevin Kloesel, for his willingness to serve on and chair my dissertation committee.
Additionally, it was the many meetings, particularly the off-site and unorthodox meetings at various baseball stadiums, that gave me just enough of a distraction to keep me (somewhat) sane.
Lastly, thanks must be given to my dissertation committee for their guidance and insights they offered me.
In particular, the guidance offered by Drs. Michael Richman and S. Lakshmivarahan went above and beyond what could reasonably be expected of a dissertation committee.


As I alluded to previously, I would not be finishing this dissertation if I had not been rescued from myself.
In addition to Dr. Kain, Drs. Michael Coniglio, David Stensrud, Harold Brooks, and Louis Wicker all played a role in creating a graduate student position for me at the National Severe Storms Laboratory.
More importantly, these scholars convinced me to undertake a dissertation project about which I was passionate, even if it was without (initially) stable funding, rather than stay in my comfort zone and pursue a dissertation project that had funding, but about which I was only tangentially interested.
It is only with the knowledge and hindsight that comes with being on the completion side of a dissertation that I realize how difficult, if not impossible, it would be to complete a dissertation on a topic about which I was less than completely passionate.


A dissertation in meteorology typically begins in childhood, when one begins to notice the beauty and power of our atmosphere.
This was certainly the case for me.
I was fortunate to have been blessed by many in my life who encouraged my pursuit of knowledge of the atmosphere.
I must thank Mr. Jay Hilgartner, Mr. Austen Onek, Mr. Ken Rank, and Mr. Garrett Lewis for encouraging me as a child to pursue meteorology; not to mention the countless hours each spent answering my questions.
Thanks are owed to the National Weather Service Forecast Office in Tulsa, Oklahoma for their continual indulgence of a wide-eyed youth from Fort Smith, Arkansas.
In particular, the willingness of Mr. Lans Rothfusz to give a father and son an impromptu private tour of the Tulsa forecast office on a Saturday afternoon and the friendship and counsel afforded me by Mr. Steven Piltz can never be repaid.


As is typically the case with most in life, my journey has been profoundly shaped by many educators along the way.
Thank you to Ms. Robin Bryan, Mr. James Moody, Mr. Charles Besancon, Mr. Larry Jones, and Dr. Barry Owen for their encouragement through my secondary education.
Thanks and gratitude must also be offered to Drs. John and Gay Stewart, my undergraduate advisors, for refusing the let me give up and for helping me reach my potential.


Friendships are important in completing a dissertation and I offer thanks to all of my friends for their friendship, of which there are too many to enumerate here.
Without your friendships I never would have remained grounded and focused on finishing my dissertation.
The friendships of three individuals in particular are most crucial in setting me on the path that has culminated with this dissertation: Mr. David Whiteis, Mr. Wayne Johnson, and Mr. Forrest Johns.
I met Mr. Whiteis at a SKYWARN Spotter training class in the spring of 1997.
The spotter class was cut short due to severe convective weather, and Mr. Whiteis offered to take me to the Fort Smith Airport to meet the local SKYWARN Spotter Network Controller, Mr. Johnson, and the local National Weather Service meteorologist, Mr. Johns.
In hindsight I often wonder what my parents were thinking to let met take a ride with a stranger to an airport to meet a bunch of other adults.
But if not for this occurrence I would have delayed, or missed out entirely on, meeting three of most influential people in my life.
Mr. Johnson, Mr. Whiteis, and Mr. Johns treated me like a son.
They nurtured my passion for severe convective weather and helped keep me on a solid foundation through my formative teenage years when I was ``too smart'' to listen to my parents.
All I can offer these three is my thanks and a promise to pay it forward.


For those unaware, hobbies that stem from science are often expensive.
I am very grateful to my family, especially my parents, Mr. Thomas Marsh and Mrs. Letitia Marsh, for their sacrifices in order to encourage my interest in the atmosphere; my passion for meteorology would have easily been extinguished if not for them.
For the many trips to the local television station to meet with Mr. Hilgartner and Mr. Onek, to the willingness to take me storm chasing to observe the atmosphere first hand, to the purchasing me countless weather instruments and books, to the plastic weather station we installed in the backyard, I cannot thank you enough.
This dissertation is as much a testament to your sacrifices, love and encouragement as it is a testament of my perseverance and work ethic --- which I learned by watching you when you did not notice.
Thank you to my brothers and sisters , aunts, uncles, and grandparents for putting up with my incessant need to talk about the weather, and all the wonderful weather gifts they have given me through the years.


Thanks must be given to my wife's family and her extended family.
Their patience with me as I completed graduate school, their prayers, and their affirmation helped sustain me through the difficult times.


And lastly, I must offer my most heartfelt and sincerest thanks to my loving wife, Sarah.
Without her love, patience, and support the dream of finishing my dissertation would have died a long time ago.
Thank you for all you did these past years, most of which I am sure I failed to recognize and appreciate.
Words can never express how much I love you.
