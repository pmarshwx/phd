%!TEX root = generalexam.tex


% BEGIN EXECUTIVE SUMMARY

The Warn-on-Forecast (WoF) initiative is tasked with establishing the scientific acumen to transform the severe convective weather warning paradigm from one based on detection to one based on prediction. The underlying idea behind WoF is to create the capability to issue warnings for severe convective hazards based on reliable, high-resolution numerical forecast guidance, instead of waiting until a severe convective hazard is observed, detected, or thought to be imminent \citep{stensrud2009wof}. Significant scientific questions must be answered and significant hurdles must be overcome before this vision can be achieved. Some of these hurdles are practical; some are theoretical; and some are even financial.


Data assimilation continues to be a source of active research for the WoF initiative. Advancements have been made from utilizing hand analyses for initializing numerical models to choosing between variational analyses methods and enesemle kalman methods.

Achieving the WoF vision is hard. Very hard. It will require significant investments of intellection capital on many fronts.


