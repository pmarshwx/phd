\documentclass{article}
\usepackage{graphicx}
\usepackage[margin=0.5in]{geometry}
\usepackage{parskip}
\setlength{\parskip}{.05in}
\usepackage{times}
\IfFileExists{revision.sty}{\usepackage{revision}}{}

\begin{document}

\section{Sources Used but Needed to be Found}
    % \begin{itemize}
    % \end{itemize}


\section{BAMS WoF Article}
    \begin{itemize}
        \item Benjamin et al. 2004 --- Observational Spacing
        \item Schmidt et al. 2002  --- Horizontal gradients in environment
        \item Wandishin et al. 2008 --- Predictability of Mesoscale Convective Systems
        \item Gong and Xu 2003 --- Aliased velocity data
        \item Liu et al. 2005 --- Biological target contamination of radar data
        \item Zhang et al. 2005 --- Biological target contamination of radar data
        \item Zhang et al. 2005 --- Initial radar QC
        \item Friedrich et al. 2006 --- Robust and fast QC of Radar Data
        \item Lakshmanan et al. 2007a --- Robust and fast QC of Radar Data
        \item Benajmin et al. 2008 --- Initial assimilation of radar data in CAMs
        \item Xue et al. 2006 --- CASA radars improve radar data in mountains
        \item Zrnic et al. 2007 --- MPAR has fast scanning for storm observations
        \item Anderson \& Collins 2007 --- Variational and Ensemble assimilation needs to be faster
        \item Gao \& Xue 2008 --- Variational and Ensemble assimilation needs to be faster
        \item Lewis et al. 2006 --- How to deal with assimilation when obs > model points
        \item Gilmore et al. 2004 --- thunderstorm simulations are sensitive to tunable parameters in single moment microphysic schemes
        \item Tong \& Xue 2008 --- thunderstorm simulations are sensitive to tunable parameters in single moment microphysic schemes
        \item Bryan et al. 2003 --- Model grid spacing effects resulting error
        \item Xue et al. 2007a --- Need ~50m grid spacing to resolve large tornadoes
        \item Ortega et al. 2009 --- SHAVE for verification purposes
        \item Erlingis et al. 2009 --- SHAVE extended to flash floods
        \item Lakshmanan et al. 2007b --- Probabilistic Hazard Information  in the warning process
        \item Kain et al. 2003 --- How to utilize the HWT in the WoF devlopment process
        \item Kain et al. 2006 --- How to utilize the HWT in the WoF devlopment process
        \item Stumpf et al. 2008 --- How to utilize the HWT in the WoF devlopment process
        \item Kong et al. 2007b --- 10-member ensemble for HWT
        \item Kuhlman et al. 2008 --- PHI in the HWT
        \item Morss et al. 2005 --- How does the public respond to warnings
        \item Kuhlman et al. 2009 --- How does the public respond to warnings
    \end{itemize}


\section{SSD Vision Document}
    \begin{itemize}
        \item What is the best way to produce 3D, high-resolution forecast fields with rapid updates?
        \item How can forecast quality be assessed and controlled in real-time?
        \item How well can this system forecast rapidly evolving and complex high-impact situations?
        \item What fields, coupled models, or uncertainty information needs to be developed to support evolving services?
        \item How to manage the seam between WoF (0--2 hrs), short-range (6-48 hrs), traditional scales (1-7 days), out into the climatological realm (weeks -- months -- years)?
        \item What is the role of the human forecaster?
    \end{itemize}

\newpage
\section{Stensrud WoF Powerpoint}
    \begin{itemize}
        \item Rapid and accurate data quality control
        \item Assimilation method to use? 3DVAR? ENKF? Hybrid?
        \item Model error: microphysics; boundary layer; turbulence; radiation; land surface
        \item Sensitivities to errors in initial conditions
        \item Ensemble methods for convective-scale
    \end{itemize}

    \begin{itemize}
        \item Xue, Droegemeier, \& Weber 2007 --- Radar Data Assimilation
        \item Dowell \& Wicker 2009 --- Radar Data Assimilation
        \item Thompson and Wicker --- Impact of Phased Array
        \item Yussouf and Stensrud 2008 --- Assimilation
    \end{itemize}




\section{Stensrud ECSS Abstract}
    \begin{itemize}
        \item Snyder \& Zhang 2003 --- Models with radar assimilation can produce reasonable forecasts; however, accurate very-short range thunderstorm forecasts are more challenging
        \item Dowell et al. 2004b --- Models with radar assimilation can produce reasonable forecasts; however, accurate very-short range thunderstorm forecasts are more challenging
        \item Tong \& Xue 2005 --- Models with radar assimilation can produce reasonable forecasts; however, accurate very-short range thunderstorm forecasts are more challenging
        \item Yssouf \& Stensrud 2010 --- Models with radar assimilation can produce reasonable forecasts; however, accurate very-short range thunderstorm forecasts are more challenging
        \item Aksoy et al. 2010 --- Models with radar assimilation can produce reasonable forecasts; however, accurate very-short range thunderstorm forecasts are more challenging
        \item Joss \& Waldvogel 1969 --- Observations demonstrate the densities and intercept parameters of hydrometeor observations can vary widely among storms and even within a single cell
        \item Prupacher \& Klett 1978 --- Observations demonstrate the densities and intercept parameters of hydrometeor observations can vary widely among storms and even within a single cell
        \item Knight et al. 1982 --- Observations demonstrate the densities and intercept parameters of hydrometeor observations can vary widely among storms and even within a single cell
        \item Ziegler et al. 1983 --- Observations demonstrate the densities and intercept parameters of hydrometeor observations can vary widely among storms and even within a single cell
        \item Cheng et al. 1985 --- Observations demonstrate the densities and intercept parameters of hydrometeor observations can vary widely among storms and even within a single cell
        \item Cifelli et al. 2000 --- Observations demonstrate the densities and intercept parameters of hydrometeor observations can vary widely among storms and even within a single cell
        \item Yussouf \& Stensrud 2011 --- Need to vary microphysics across ensemble members when looking at extreme events within the ensemble
        \item Smith et al. 2010 --- 3DVAR in the HWT
        \item Stensrud et al. 2010 --- 3DVAR in the HWT
        \item Stensrud \& Gao 2010 --- Environmental conditions affect the quality of the storm-scale analyses and forecasts

    \end{itemize}

\end{document}