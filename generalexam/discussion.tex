%!TEX root = generalexam.tex

% BEGIN CHAPTER

Achieving WoF will be hard. Very hard. It \emph{is} hard. It will not happen overnight, but then again most things that are worth anything are worth working hard toward and waiting for.  No one individual will have all the answers and WoF will require a significant amount of collaboration amongst the many players.


This document presents some information regarding challenge facing WoF, but it only begins to scratch the surface.The American economy is still in the doldrums and Government is wanting to slash science funding.  Decreasing budgets for science, and in turn observing networks, will make establishing observational networks to aid a WoF system harder to afford.  Can a WoF system provide near continuous updates of high-resolution, calibrated, probabilistic information without having near continuous updates of high-resolution observations? If these observations are not going to be available, then add yet another hurdle WoF will need to overcome.


Questions also remain as to how forecasters will utilize probabilistic information in an operational setting.  The sheer amount of information that forecasters will be required to synthesize is unfathomable at this time.  The author remembers participating in an HWT-Experimental Warning Program (EWP) experiment investigating Phased Array Radar output. After examining what felt like hours worth of data, participants were informed they only has examined 20-30 minutes worth. Imagine a warning operations setting in which Phased Array Radar output with O(1 min) updates are being thrust upon a forecaster, combined with near real-time analyses from a multitude of ensembles.


WoF will require a significant upgrade to infrastructure and bandwidth. As is stands now, personal communication with operational forecasters in local weather service offices complain that the bandwidth is not sufficient to acquire all of the information available now.  What happens when a WoF system, and the (hopefully) high-resolution observations that come with WoF, increases the amount of information by 10 fold or more?


Also, WoF would be remiss if it did not at least begin to think about how to convey this new, probabilistic information to a public who has an attention span with weather that waxes and wanes depending on the type of year.  How will people take to receiving probabilistic information in the context of severe convective hazards? Granted, probabilistic information will allow for individuals to compute personal cost/loss rations and determine for themselves what their personal thresholds of certainty need to be, but who is going to teach them to use WoF information in this manner? Meteorologists of the future will have to educate and communicate this information to a wide ranging of audiences ranging from the public at-large to emergency managers, to highly educated users (aka, other meteorologists). Future meteorologists will need to understand and be ready to trouble shoot data issued with a WoF system, in addition to actually knowing meteorology.  Meteorologists of the future will need to be a teacher, communicator, statistician, computer scientist, and meteorologist, all-in-one. Current academic curricula are not necessarily conducive to this, and making changes to curricula is no small task.


With all of what has been written here -- the discussion of the potential challenges and pitfalls -- the author is excited about the future of convective scale meteorology. Even if WoF is unobtainable in the near future, the number of advancements that will emanate from this initiative is northing short of a scientific revolution. It is an exciting time to be a convective scale meteorologist.