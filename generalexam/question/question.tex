\documentclass{article}
\usepackage{graphicx}
\usepackage[margin=0.5in]{geometry}
\usepackage{parskip}
\setlength{\parskip}{.2in}
\usepackage{times}

\begin{document}

\title{Patrick Marsh's General Examination}
\date{30 January 2012 @ 1800 UTC --- 29 February 2012 @ 1800 UTC}
\author{Kevin Kloesel, Chair\\
        John Kain, Co-Chair\\
        Frederick Carr, Member\\
        Michael Richman, Member\\
        David Stensrud, Member\\
        S. Lakshmivarahan, Outside Member\\}

\maketitle

\bigskip\bigskip

The convective-scale ``Warn-on-Forecast'' paradigm (WoF) includes explicit numerical prediction of hazardous weather elements such as tornadoes, downbursts, and large hail. Current operational numerical modeling systems are not designed for the time and space scales on which these phenomena typically occur. Furthermore, the inherent predictability of these phenomena is not firmly established. These limitations alone present significant challenges for the WoF initiative. For the written portion of your general examination, provide an assessment of the primary scientific challenges facing the WoF effort, including a detailed summary of the theoretical and practical obstacles in these specific components of the WoF program:


    \begin{enumerate}
        \setlength{\parskip}{.025in}
        \item Assimilation of radar data (radial velocity and reflectivity), including the basis for and strengths/weaknesses of 3DVAR, 4DVAR, EnKF, and others, including hybrid approaches;
        \item Error growth and predictability on the convective scale; and
        \item Development, testing, optimization, and application of convective-scale ensembles
    \end{enumerate}

For this task, information is to be gathered from books, journals, and online resources, but not from personal communication with scientists currently working on the WoF project. All sources of information are to be documented as they would be for the scientific literature. If you need further clarification on this assignment, please contact your Committee Co-Chair. Your written summary will be due one month from the date you receive this letter. Please submit the summary to the entire committee as a PDF file.

\end{document}