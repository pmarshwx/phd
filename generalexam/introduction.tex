%!TEX root = generalexam.tex


% BEGIN CHAPTER

The Warn-on-Forecast (WoF) initiative is tasked with establishing the scientific acumen to transform the severe convective weather warning paradigm from one based primarily on detection to one based more on prediction. The underlying idea behind WoF is to create the capability to issue warnings for severe convective hazards based on reliable, high-resolution numerical forecast guidance, instead of waiting until a severe convective hazard is observed, detected, or thought to be imminent \citep{stensrud2009wof}. However, significant hurdles must be overcome for this vision to become a reality.


The prospect of fundamentally shifting the prevailing paradigm in the convective-scale numerical modeling community, such as WoF proposes doing, is not without precedent. In 1990, Douglas Lilly challenged the status quo by rhetorically asking if the time had come for cloud-resolving scale forecasts of a predictive ability \citep{lilly1990}. Lilly recognized this would not be easy. He discussed numerous scientific hurdles, but stressed that through collaboration across agencies; renewed emphasis on predictability studies; and improvements in observations, data assimilation and model parameterizations, the challenge was surmountable in the coming decade --- and he was right! Operational, semi-operational, and research centers now regularly produce contiguous United States (CONUS) or near-CONUS scale numerical forecasts with grid spacing \mbox{O(1 km)} without the use of convective parameterization schemes. More importantly, these forecasts produce useful guidance of convective initiation (preliminary results from the 2011 HWT Experimental Forecast Program), reasonably accurate depictions of convective evolution (e.g., \citealp{clark2012overview, clark2010mcv, clark2010verification, clark2009comparison, schwartz2009camresolution, kain2010attributes, kain2008camconsiderations, done2004cams}), and improved forecasts of precipitation \citep{novak2011qpf}.


Fast forward 20 years and the convective hazards research community is once again being pushed to alter how it approaches the prediction of high-impact convective hazards -- this time through the WoF initiative. Many of the challenges laid forth by \cite{lilly1990} will once again have to be reexamined. Although the realm of data assimilation has experienced rapid advancements, determining the initial states of numerical models remains a significant challenge; studies on predictability and error growth will help to frame the WoF vision by placing bounds on reasonable expectations; and discussions on convective-scale ensembles will frame how WoF is implemented and the resources needed for it to achieve fruition. The WoF initiative will require close collaboration across research organizations; it will push the bounds of computational resources; and will require the development of new approaches to tackle old problems -- especially the need for efficient numerical techniques involved in data assimilation.






